\documentclass{scrreprt}
\usepackage{psfig}

\makeatletter
\makeatother

\begin{document}

\title{PuMa-II Software Specification}

\author{W.H.B.\ van Straten, B.W.\ Stappers, and K.\ Ramesh}

\maketitle
\tableofcontents{}

\begin{abstract}
This document specifies the software element of the 2nd generation of
the Dutch Pulsar Machine, PuMa-II, to be installed at the Westerbork
Synthesis Radio Telescope, WSRT, in 2005.  PuMa-II will consist of
eight data acquisition machines, called the \emph{Primary} nodes, and
a processing cluster of \emph{Secondary} nodes.

Each Primary Node will interface with one of the 20\, MHz, 12-bit
sampled outputs of the WSRT tied-array beam former.  On each machine,
a custom-designed interface card (PiC) will re-sample the data with
8-bit resolution and write the data to computer RAM via a direct
memory access (DMA) interface.  A large file storage facility,
implemented as internal RAID, will be used to store data locally, and
a high-speed interconnect will be used to send the data to Secondary
nodes for data reduction.

The communications plan and protocol used for sending the data from
Primary to Secondary node may change over the life of the PuMa-II
instrument.  Therefore, an adaptable and modular design philosophy has
been employed in the data flow control software, based on the use of a
ring buffer in shared memory.  This document contains an overview of
the software model, including a discussion of the design decisions
made at each level of command, control, and configuration.

\end{abstract}


\chapter{Introduction}

\section{Design Goals}

\begin{enumerate}
\item modularity
\vspace{-2mm}
\item separation of data transport, control, and data reduction
\vspace{-2mm}
\item near real-time data reduction
\vspace{-2mm}
\item elegant scaling to slower data reduction speeds
\end{enumerate}

\section{Overview}

\subsection{Workstation Cluster}

The functionality of PuMa-II will be divided across multiple workstations.
These are divided into two main classes:

\begin{enumerate}

\item Primary nodes - workstations equipped with Direct Memory Access 
	(DMA) card, large data storage, and high-speed interconnect.

\item Secondary nodes - workstations equipped with high-speed interconnect
	and modest data storage facilities.

\end{enumerate}

A single Primary node may have multiple Secondary nodes associated
with it.  To these, it will send the observed data either during an
offline, post-recording stage or in quasi-realtime.  The Secondary
nodes will contain a large amount of RAM and a moderate internal data
storage medium.

\begin{figure}
\centerline{\psfig{figure=layout.eps,width=3.5in,angle=0}}
\caption [\sffamily PuMa-II Data Flow]
{
Schematic overview of PuMa-II data flow; note the parallelism and 
symmetry of the operations on Primary and Secondary nodes.
}
\label{fig:layout}
\end{figure}

\subsection{Software}

The major functionality of PuMa-II is divided into five categories:

\begin{enumerate}

\item Data Flow Control - high-bandwidth data transfer and storage
\item Data Reduction - process and archive the observed data
\item Command and Monitoring - communication channels for external control
\item Control Interface - centralized access to Command and Monitoring
\item Configuration and Scheduling - files and databases for automated control

\end{enumerate}

Communication between each of the levels of software will take place
primarily through shared memory, semaphores, and internet socket
connections.

\subsubsection{Data Flow Control}

Data Flow Control software will control all aspects of the
high-bandwidth data recording, including DMA control, interim storage
on internal RAID, and high-speed transfer between the Primary and
Secondary nodes.  The transfer to Secondary nodes will include the
overlap required to compensate for data reduction losses (owing to
dispersion smearing, filter rise times, etc.) as specified by the
Configuration and Scheduling software.

On both Primary and Secondary nodes, data will be stored in a large
buffer established as shared memory, called the Data Block.  The
various tasks that must run in parallel will be implemented as unique
processes, as opposed to multiple threads of a single process.
Therefore, access to the Data Block will be controlled by an
inter-process communication (IPC) locking method, known as a
semaphore.

Rational: It is better to begin with multiple processes and IPC in the
early stages of development because this paradigm is more modular.
For example, the process that reads data from the Data Block and
writes it to local file storage may be run on either a Primary or
Secondary node.  If data reduction can later be performed in
real-time, the disk writing client may be replaced by a data
processing client.

\subsubsection{Data Reduction}

On both Primary and Secondary nodes, one or more Data Clients may
attach to the Data Block and operate on the data.  A single,
high-priority Data Client will be given permission to flag sub-blocks
as processed.  Initially, this client will be part of the Data Flow
Control software that writes the data to local file storage.  Later,
this client may be a data processing client.  

Data Clients may perform any number of tasks, including various forms
of data reduction, calculation and display of data quality statistics
such as the bandpass, storage of the data to some form of medium, or
farming the data out to a grid.  The data reduction client will
process the data according to the specifications of the Configuration
and Scheduling software.

\subsubsection{Command and Monitoring}

Command and Monitoring software includes the Command software that
establishes low-bandwidth network communication channels between
Primary and Secondary nodes and the Control Interfaces.  These
channels are used for sending high-level control commands, such as
start, stop, and information about the source.  These communication
channels may be implemented as a control thread in each component of
the Data Flow Control software. 

The Monitoring software will perform any tasks required to maintain
proper operation of the instrument and present useful information to
the user.  This includes monitoring telescope status information from
TCS, disk space consumption, network traffic, CPU load, etc.

\subsubsection{Control Interface}

The Control Interface software defines the centralized command/control
point, and will be connected to the various communication channels
established by the Control and Monitoring software on each of the
Primary and Secondary nodes.  The Control Interface software should be
run on a single designated workstation, as it will provide the means
through which other processes may treat PuMa-II as a single
instrument.  For example, a text or graphical user interface and/or
automated scheduling program may connect to PuMa-II, issue commands,
and inquire about the status of the instrument.  A textual user
interface (TUI) will be developed to connect to the Control Interface
and:
\begin{itemize}
\item allow PuMa-II to be configured, started, and stopped;
\vspace{-2mm}
\item display various status variables; and
\vspace{-2mm}
\item create diagnostic plots, such as the passband and digitizer statistics
\end{itemize}

\subsubsection{Configuration and Scheduling}

Configuration and scheduling software will be make use of database
information to configure the PuMa-II instrument and schedule data
reduction operations, based upon the parameters of the observation.

\chapter{Data Flow Control Software}

Data Flow Control software running on the Primary and Secondary nodes
must handle the flow of data in a modular and extensible manner,
enabling future developments by replacement of a single component.
The required modularity is met by basing all data transfer on a single
ring buffer protocol, which will be known as the Data Block.

\section{Data Block}
\label{sec:data_block}

The Data Block is a ring buffer that will be allocated as a shared
memory resource, logically divided into a header block followed by a
number of sub-blocks.  Each sub-block will have an associated byte
count that may be used to calculate the time offset from the start of
the observation.  Only one contiguous stream of data may be
represented in the Data Block at any one time; therefore, the size of
the Data Block will determine the maximum amount of time required to
flush the ring buffer between stopping an observation and starting the
next observation.

%\subsubsection{Start of Data}

At the beginning of an observation, every sub-block of the ring buffer
will be empty and the header block will be initialized with the
relevant observation information (such as bandwidth, centre frequency,
source, start time, etc.). In order that data acquisition may be
started before the data are valid, data may be written to sub-blocks
before the start-of-data flag is raised.  Data may be read from
sub-blocks only after the start-of-data flag is raised.

%\subsubsection{End of Data}

Data will be written to sub-blocks in sequential order until the end
of the observation, at which point an end-of-data flag will be raised,
the last full sub-block and the number of valid bytes written to this
sub-block will be recorded.   Data will be read from sub-blocks in
sequential until the end-of-data flag is raised and the current
sub-block is equal to the last full sub-block.

%\subsubsection{Data Validity}

Data that are written to the Data Block may not necessarily be valid.
Therefore, each sub-block will have associated variables to indicate:
\begin{itemize}
\item the state of the block: empty or full.
\item the byte offset at which data became valid
\item the byte offset at which data became invalid
\end{itemize}
Note that data can transit from valid to invalid (or vice versa) only
once per data block.  However, these transitions may occur an
arbitrary number of times between the start and end of data.

\subsection{Write Client}

A single, high-priority process, called the Write Client, will be
given write access to the Data Block; only the Write Client can change
the state of a sub-block from empty to full.  The Write Client can
write data to the ring buffer before flagging the start of the
observation.  In this way, it can clock data without activating the
Read Clients, and may change the state of the validity flags before
raising the start-of-data flag.

After raising the start-of-data flag, the Write Client will not write
data to a sub-block until its state is empty; after filling a
sub-block, it will change its state to full and set the data validity
byte offsets.  If the Write Client cannot obtain an empty sub-block,
an overflow condition will occur; this condition will be handled
according to the mode of operation:

\begin{itemize}

\item {\bf contiguous}, a contiguous stream of data is required (as is
  often the case in search observations).  In this case, data overflow
  is treated as an error that is propagated to the Command and Control
  software, and data acquisition is stopped.

\item {\bf discontinuous}, an end-of-data is written to the Data
  Block, and the Write Client continues to receive data from its
  source.  When the ring buffer has been cleared by the Read Client,
  the Write Client starts a new observation and data acquisition
  continues.

\item {\bf tolerant}, the Write Client will wait indefinitely for
  the next empty sub-block.

\end{itemize}

Other specifics of overflow handling will depend upon the application.
For example, discontinuous overflow handling may include a sending a
signal to to the Primary Node that instructs it to move on to the next
Secondary node.

\subsection{Read Clients}

One or more Read Clients may attach to the Data Block and read the
data from sub-blocks marked as full.  Only the bytes designated as
valid will be used from each sub-block.  Only a single, high-priority
Read Client will be given permission to change the state of a
sub-block from full to empty.  Read Clients will access sub-blocks in
contiguous order after the start-of-data flag is raised and until the
end-of-data condition is encountered.

\subsubsection{Example}

Consider a tight schedule, in which the time required to synchronize
and start data acquisition is considered too costly.  In this case,
some time may be saved by starting data acquisition before the
telescope is on source and continuing to acquire data while the
telescope slews between sources.

In this case, the Write Client will begin writing data to the Data
Block before raising the start-of-data flag.  The Write Client knows
the UTC time at which it started clocking data into the Data Block.
When the signal is given to the Write Client that the data became
valid at a certain UTC time, the Write Client can go back to that time
in its buffer, flag all data from that point to the present buffer as
valid, and raise the start-of-data flag.  In this way, the Write
Client can retroactively flag data as valid {\bf before the start of
the observation}.

In order to slew to the next source, a data invalid message must be
sent to the Write Client before the data becomes invalid.  The Write
Client will continue to clock data into the Data Buffer, but the data
will be marked as invalid and therefore will be ignored by the Read
Clients; the designated Read Client will simply mark the buffers as
empty as they are encountered. (This is why retroactive validity
flagging can be done only before the start of the observation.)

\section{Data Flow Write Clients}

Write Client software will read data from a device and write it to
the Data Block.

\subsection{DMA Client: {\tt dma2db}}

The DMA Client software, {\tt dma2db}, is responsible for transferring
data from the telescope to the Data Block.  It will talk directly to
the PiC through its PCI interface, start and stop the data transfer,
and record the UTC start time of the observation.  Data from the PiC
will be transferred to Primary node RAM via a Direct Memory Access
(DMA) card that is commercially available from Engineering Design Team
(EDT).  The DMA Client software will:

\begin{enumerate}

\item allocate a number of fixed memory buffers of a size and number
to be determined during the testing stage;

\item send start and stop instructions to the PiC via the PCI/DMA interface;

\item determine the UTC start time of the first sample recorded

\item copy filled DMA buffers to the Data Block; and

\item monitor the number of DMA buffers filled and copied, ensuring that
no data overflow occurs.

\end{enumerate}

\noindent
The DMA buffers will be separate from the Data Block buffers and
accessed only by the DMA card driver and {\tt dma2db}.  Once started,
DMA transfer will continue uninterrupted until a stop flag is raised
or an overflow occurs.

\subsection{Network Interface Client: {\tt nic2db}}

The software for network I/O will run on both Primary and Secondary
nodes.  The Data Flow Control software running on the Secondary nodes,
{\tt nic2db}, will open a port and listen for incoming Data Flow
Control connections from {\tt db2nic}, which will run on the Primary
nodes.  A single incoming channel will be connected and used to
establish a high-bandwidth data communication channel between a single
Primary node and a single Secondary node.  The protocol for the
network communications will be a simple, custom-built design on top of
internet sockets.  This may change in the future to some sort of
grid-based protocol.  Data received via this communication channel
will be copied to the Data Block in contiguous order.  Each packet of
data will be preceded by a copy of the Data Block header from the
Primary node.  This header will be copied to the Secondary node Data
Block.

The {\tt nic2db} software has the responsibility to monitor the Data
Block and ensure that there is sufficient space to hold incoming data
packets.  It will send a message to the Primary node if there is
insufficient space, and the Primary node will cease data transfer,
possibly initiating data transfer to the next in Secondary node in the
ring.

\section{Data Flow Read Clients}

Read Client software will read data from the Data Block and write it
to a device.

\subsection{Data Storage Client: {\tt db2disk}}

Writes data blocks to disk, breaking up data into files of arbitrary
length.  Each file will be preceded by the header block from the Data
Block.  Runs on either Primary or Secondary nodes, depending on the
mode of operation.  After each file is written to disk, an entry will
be added to an ASCII text log file; each entry will describe:
\begin{itemize}
\item the full path to the file
\item the time it was written
\item the size of the file
\item the time required to write the file
\item the observation identifier
\end{itemize}
This log file will be polled by the Configuration and Scheduling
software, which will add the information to a centralized database.

\subsection{Network Interface Client: {\tt db2nic}}

This software runs on the Primary nodes; it reads from the Data Block
and writes to one or more Secondary nodes, breaking up data into
packets of arbitrary length.  The total length of data sent to an
individual Secondary node will be independent of the Data Block buffer
sizes, and may depend on the overlap specified by the Configuration
and Scheduling software.  Header information (including all available
observation information as well as offset byte counts) will be sent
with each block of data transmitted to the Secondary nodes.

\chapter{Data Reduction Software}

\section{Operational Phases}

The data reduction software supports three models:

\begin{itemize}
\item {\bf offline} processing: performed after the recording has finished
\item {\bf simultaneous} processing: performed during the recording
\item {\bf diskless} processing: as above, but completely in memory
\end{itemize}

\subsection{Offline Data Reduction}

The data reduction software operates only on data files stored on
local disk space, and only after the recording has completed.  After
the observations have been completed, the data files from each Primary
node will be farmed out to the Secondary nodes.  After each file has
been copied, an entry will be added to the log file that is monitored
by the Configuration and Scheduling software, which will attend to the
data reduction as described in Chapter 6.

\subsection{Simultaneous Data Reduction}

The data reduction software runs during the observation.  In this
mode, data are sent directly to Secondary nodes and written to disk.
As before, an entry will be made in a log file and the Configuration
and Scheduling software will coordinate the data reduction.

\subsection{Diskless Data Reduction}

The data reduction software is able to keep up completely with the
flow of data on the Primary nodes.  In this case, the data is never
written to disk, and {\tt dspsr} operates as a Read Client directly
connected to the Data Block.  Another Read Client may be written that
will farm data reduction tasks to a grid computing facility.  In this
case, the data reduction Read Client will have to incorporate the
Configuration and Scheduling tasks.


\chapter{Command and Monitoring Software}

\section{Command}

Each element of the DADA software must be coordinated to operate as
a single instrument.  Therefore, many of the processes described in
the previous chapter will have to be synchronized and configured
through communications channels.  Some degree of synchronization will
be achieved through the hand-shaking protocol of the Data Block
specification.  Other communication requirements will be met through
internet socket connections.

\subsection{Data Block Communications}

Many of the Data Clients can be implemented as a persistent process,
like a daemon, that is configured once during an initialization stage
and runs automatically from that point onward.  Apart from
configuration, the behaviour of these automatic processes will depend
completely upon the state of the Data Block.

Two Read Client programs that can run in this manner are {\tt dbdisk}
and {\tt dbnic}.  These automatic processes need only start reading
from the Data Block when it is active, as determined by the behaviour
of the Write Client.  They start when the header is initialized and
stop when the end of data flag is raised.
Also, the {\tt nicdb} Write Client can be run as an automatic process
that starts when it receives packets from the Primary Node and ends
when the end of data message is received.
If it is shown that the operation of these Data Clients may have to
change from observation to observation, then there are two possible
solutions:
\begin{itemize}
\item Stop and restart the daemons with different configuration parameters
\item Enable socket communications that set configuration parameters
\end{itemize}

\subsection{Internet Socket Communications}

Certain processes will require internet socket communications in order
to be configured between observations and to start and stop the
observation.  In order that communications may be sent and received
during normal operation, the processes that require socket
communications will be multi-threaded.  The communication threads may
have lower priority than the main thread, if required.  More than one
communication channel may be opened to each process; however, only one
channel will be able to issue control commands.  The others may only
inquire about the status of the process.

All communications will be human readable, ASCII text.  This enables
interface testing using standard tools like telnet.  If large amounts
of binary data must be sent between the Control Interface and Data
Flow Control software, then it should be done using a separate
communication channel designated for this purpose.  Text commands will
be sent on a single line of text.  After every command received, the
process will respond with {\tt ok} or {\tt fail}, followed by any
additional information, and ending with the command prompt.

\subsubsection{DMA Data Client}

The {\tt dmadb} processes running on each of the primary nodes
require careful synchronization with the Control Interface software,
especially if they are all to be started on the same UTC second.  For
this reason, it will not suffice to remotely start the processes at
the desired beginning of each observation.  The processes must be
persistent and must maintain socket communications with the Control
Interface.  The Command Interface to {\tt dmadb} is described in
Section~\ref{sec:pwc}.

\section{Monitoring}

Monitoring processes will be run on all nodes in the instrument,
reporting on remaining disk space, CPU load, network traffic, etc.  At
the time of this writing, it is not clear if standard cluster
monitoring tools like Ganglia will suffice.  For example, it may prove
useful to have a regular report on which Secondary nodes are currently
receiving data.  This information would have to come from {\tt dbnic},
possibly via a socket connection to this process.

In addition to live monitoring, it may also prove useful to maintain a
database of relevant statistics, such as the average time required to
write a block of data to file or over the network.  These monitoring
tasks would be performed by the relevant process, {\tt dbdisk} and/or
{\tt dbnic} and communicated to a central database via some protocol.

\chapter{Control Interface Software}


\chapter{Configuration and Scheduling Software}

Depending upon the objectives of the experiment, DADA configuration
may change from observation to observation.

\section{Instrumental Configuration}

Based upon the parameters of observation, the {\tt dada} control
program will set up various operational parameters, including:
\begin{itemize}
\item buffer sizes: DMA, Data Block, file and network I/O
\item number of Primary nodes
\item number of Secondary nodes
\item assignment of Secondary to Primary nodes
\item operational mode: offline, simultaneous, or diskless
\item overlap required between Secondary nodes
\end{itemize}
These operational parameters will be stored in a {\bf configuration
database}, which will specify the instrument configuration and data
reduction requirements for various combinations of source, receiver,
centre frequency, band width, etc.

Entries in the configuration database may have an expiration date
associated with them.  In this manner, the observer may specify
special configuration and/or reduction options for a specific
experiment without permanently changing the default behaviour.

\section{Data Reduction}

The Configuration and Scheduling program, {\tt dadaskd}, will be used
to configure and schedule all data reduction operations.  Before the
{\bf diskless} mode of data reduction is implemented, all data will
exist as a file on either the Primary or Secondary nodes.  Whenever a
file is written to disk, an entry will be registered in a centralized
{\bf observations database}, which will contain basic header
information such as
\begin{itemize}
\item source name

\item start time (UTC)

\item centre frequency (MHz)

\item band width (MHz)
\end{itemize}
as well as the location (machine and file name) of the data.  Each
entry will also contain a time-stamped list of {\bf performed
operations}, describing when the data was written, when and how it was
processed, when it was deleted, etc.  The header information will be
used to find matches in the configuration database.  An observation
may be processed in multiple ways, as specified by the list of {\bf
requested operations} in the configuration database entry.

If it is possible to achieve two requested operations in one execution
of {\tt dspsr} then this will be done.  Otherwise, data reduction
operations will be performed one at a time.  After each operation is
completed, it will be recorded in the list of performed operations of
the observations database entry.

The scheduling software will periodically check or poll the
observations database.  Any entries that require data reduction will
be scheduled according to the data reduction parameters of the
requested operations in the corresponding configuration database
entry.  An observation will be considered completely processed when
the list of performed operations is equal to the list of requested
operations.  At this point, the raw data will be deleted or archived.

\chapter{Summary}

The following table summarizes the software that will be developed,
its basic functionality, and the machine on which it will run.

\vspace{5mm}

\begin{tabular}{l|p{8cm}|l}

Name & \multicolumn{1}{c}{Function} & Machine \\ \hline

{\tt dma2db} & Transfers data from EDT buffers to the Data Block.
	& Primary \\

{\tt db2disk} & Reads data from the Data Block and writes it, with
	header information, to disk. & Both \\

{\tt db2nic} & Reads data from the Data Block and sends it, with
	header information, to a Secondary node. & Primary \\

{\tt nic2db} & Receives data from Primary node and writes it to the 
	Data Block. & Secondary \\

{\tt dspsr} & Attaches to the Data Block and processes raw data
	according to specification, writing results to disk. &
	Secondary \\

{\tt puma2} & Connects to the Command and Control interface of the
	various Primary nodes.  Accepts and maintains external
	text-based control connection.  & fixed \\

{\tt puma2tms} & Translates TMS control structure packets into
	text-based commands as accepted by {\tt puma2} control
	connection. & fixed \\

{\tt puma2tui} & Textual user interface connects to {\tt puma2},
	displays various quantities of interest, and controls
	the instrument. & variable \\

{\tt puma2skd} & High-level code that schedules and records
        data reduction operations & fixed \\

\end{tabular}



\end{document}
