\documentclass{scrreprt}
\usepackage{psfig}
\usepackage{atbeginend}

\makeatletter
\makeatother

\BeforeBegin{itemize}{\setlength{\topsep}{-2mm}}
\BeforeBegin{itemize}{\setlength{\partopsep}{-2mm}}

\BeforeBegin{enumerate}{\setlength{\topsep}{-2mm}}
\BeforeBegin{enumerate}{\setlength{\partopsep}{-2mm}}

\AfterBegin{itemize}{\setlength{\itemsep}{-1mm}}
\AfterBegin{enumerate}{\setlength{\itemsep}{-1mm}}

\begin{document}

\title{Distributed Acquisition and Data Analysis (DADA) Software Specification}

\author{Willem van Straten}

\maketitle
\tableofcontents{}

\begin{abstract}

This document describes the software component of the Distributed
Acqusisition and Data Analysis (DADA) system.  This software has been
developed to support the recording and processing of timeseries from
one or more digitizers using one or more computers.  The first version
of the code was written for the 2nd generation of the
Caltech-Parkes-Swinburne Recorder, CPSR-II, installed at the Parkes
Observatory in 2002.  This code was updated and revised for the 2nd
generation of the Dutch Pulsar Machine, PuMa-II, installed at the
Westerbork Synthesis Radio Telescope (WSRT) in 2005.  The next
revision of this code will be used to implement the
ATNF-Parkes-Swinburne Recorder, APSR, to be installed at Parkes in
2007.

The DADA software must be run on one or more data acquisition
machines, called the \emph{Primary} nodes, and may optionally include
the use of a (possibly distributed) processing cluster of
\emph{Secondary} nodes.

Each Primary Node interfaces directly with the data acquisition
hardware at the telescope; for example, via a Direct Memory Access
(DMA) card, a PCI card, or a Network Interface Card (NIC).  From RAM,
the Primary Node may write the data to a local file system, forward
the data on to one or more Secondary nodes, or process the data
directly from RAM.

The communications plan and protocol used for sending the data from
Primary to Secondary node may change over the life of an instrument.
Therefore, an adaptable and modular design philosophy has been
employed in the data flow control software, based on the use of a ring
buffer in shared memory.  This document contains an overview of the
software model, including a discussion of the design decisions made at
each level of command, control, and configuration.

\end{abstract}


\chapter{Introduction}

\section{Design Goals}

\begin{enumerate}
\item modularity
\vspace{-2mm}
\item separation of data transport, control, and data reduction
\vspace{-2mm}
\item near real-time data reduction
\vspace{-2mm}
\item elegant scaling to slower data reduction speeds
\end{enumerate}

\section{Overview}

\subsection{Workstation Cluster}

The functionality of PuMa-II will be divided across multiple workstations.
These are divided into two main classes:

\begin{enumerate}

\item Primary nodes - workstations equipped with Direct Memory Access 
	(DMA) card, large data storage, and high-speed interconnect.

\item Secondary nodes - workstations equipped with high-speed interconnect
	and modest data storage facilities.

\end{enumerate}

A single Primary node may have multiple Secondary nodes associated
with it.  To these, it will send the observed data either during an
offline, post-recording stage or in quasi-realtime.  The Secondary
nodes will contain a large amount of RAM and a moderate internal data
storage medium.

\begin{figure}
\centerline{\psfig{figure=layout.eps,width=3.5in,angle=0}}
\caption [\sffamily PuMa-II Data Flow]
{
Schematic overview of PuMa-II data flow; note the parallelism and 
symmetry of the operations on Primary and Secondary nodes.
}
\label{fig:layout}
\end{figure}

\subsection{Software}

The major functionality of PuMa-II is divided into five categories:

\begin{enumerate}

\item Data Flow Control - high-bandwidth data transfer and storage
\item Data Reduction - process and archive the observed data
\item Command and Monitoring - communication channels for external control
\item Control Interface - centralized access to Command and Monitoring
\item Configuration and Scheduling - files and databases for automated control

\end{enumerate}

Communication between each of the levels of software will take place
primarily through shared memory, semaphores, and internet socket
connections.

\subsubsection{Data Flow Control}

Data Flow Control software will control all aspects of the
high-bandwidth data recording, including DMA control, interim storage
on internal RAID, and high-speed transfer between the Primary and
Secondary nodes.  The transfer to Secondary nodes will include the
overlap required to compensate for data reduction losses (owing to
dispersion smearing, filter rise times, etc.) as specified by the
Configuration and Scheduling software.

On both Primary and Secondary nodes, data will be stored in a large
buffer established as shared memory, called the Data Block.  The
various tasks that must run in parallel will be implemented as unique
processes, as opposed to multiple threads of a single process.
Therefore, access to the Data Block will be controlled by an
inter-process communication (IPC) locking method, known as a
semaphore.

Rational: It is better to begin with multiple processes and IPC in the
early stages of development because this paradigm is more modular.
For example, the process that reads data from the Data Block and
writes it to local file storage may be run on either a Primary or
Secondary node.  If data reduction can later be performed in
real-time, the disk writing client may be replaced by a data
processing client.

\subsubsection{Data Reduction}

On both Primary and Secondary nodes, one or more Data Clients may
attach to the Data Block and operate on the data.  A single,
high-priority Data Client will be given permission to flag sub-blocks
as processed.  Initially, this client will be part of the Data Flow
Control software that writes the data to local file storage.  Later,
this client may be a data processing client.  

Data Clients may perform any number of tasks, including various forms
of data reduction, calculation and display of data quality statistics
such as the bandpass, storage of the data to some form of medium, or
farming the data out to a grid.  The data reduction client will
process the data according to the specifications of the Configuration
and Scheduling software.

\subsubsection{Command and Monitoring}

Command and Monitoring software includes the Command software that
establishes low-bandwidth network communication channels between
Primary and Secondary nodes and the Control Interfaces.  These
channels are used for sending high-level control commands, such as
start, stop, and information about the source.  These communication
channels may be implemented as a control thread in each component of
the Data Flow Control software. 

The Monitoring software will perform any tasks required to maintain
proper operation of the instrument and present useful information to
the user.  This includes monitoring telescope status information from
TCS, disk space consumption, network traffic, CPU load, etc.

\subsubsection{Control Interface}

The Control Interface software defines the centralized command/control
point, and will be connected to the various communication channels
established by the Control and Monitoring software on each of the
Primary and Secondary nodes.  The Control Interface software should be
run on a single designated workstation, as it will provide the means
through which other processes may treat PuMa-II as a single
instrument.  For example, a text or graphical user interface and/or
automated scheduling program may connect to PuMa-II, issue commands,
and inquire about the status of the instrument.  A textual user
interface (TUI) will be developed to connect to the Control Interface
and:
\begin{itemize}
\item allow PuMa-II to be configured, started, and stopped;
\vspace{-2mm}
\item display various status variables; and
\vspace{-2mm}
\item create diagnostic plots, such as the passband and digitizer statistics
\end{itemize}

\subsubsection{Configuration and Scheduling}

Configuration and scheduling software will be make use of database
information to configure the PuMa-II instrument and schedule data
reduction operations, based upon the parameters of the observation.

\chapter{Data Flow Control Software}

Data Flow Control software running on the Primary and Secondary nodes
must handle the flow of data in a modular and extensible manner,
enabling future developments by replacement of a single component.
The required modularity is met by basing all data transfer on a single
ring buffer protocol, which will be known as the Data Block.

\section{Data Block}
\label{sec:data_block}

The Data Block is a ring buffer that will be allocated as a shared
memory resource, logically divided into a header block followed by a
number of sub-blocks.  Each sub-block will have an associated byte
count that may be used to calculate the time offset from the start of
the observation.  Only one contiguous stream of data may be
represented in the Data Block at any one time; therefore, the size of
the Data Block will determine the maximum amount of time required to
flush the ring buffer between stopping an observation and starting the
next observation.

%\subsubsection{Start of Data}

At the beginning of an observation, every sub-block of the ring buffer
will be empty and the header block will be initialized with the
relevant observation information (such as bandwidth, centre frequency,
source, start time, etc.). In order that data acquisition may be
started before the data are valid, data may be written to sub-blocks
before the start-of-data flag is raised.  Data may be read from
sub-blocks only after the start-of-data flag is raised.

%\subsubsection{End of Data}

Data will be written to sub-blocks in sequential order until the end
of the observation, at which point an end-of-data flag will be raised,
the last full sub-block and the number of valid bytes written to this
sub-block will be recorded.   Data will be read from sub-blocks in
sequential until the end-of-data flag is raised and the current
sub-block is equal to the last full sub-block.

%\subsubsection{Data Validity}

Data that are written to the Data Block may not necessarily be valid.
Therefore, each sub-block will have associated variables to indicate:
\begin{itemize}
\item the state of the block: empty or full.
\item the byte offset at which data became valid
\item the byte offset at which data became invalid
\end{itemize}
Note that data can transit from valid to invalid (or vice versa) only
once per data block.  However, these transitions may occur an
arbitrary number of times between the start and end of data.

\subsection{Write Client}

A single, high-priority process, called the Write Client, will be
given write access to the Data Block; only the Write Client can change
the state of a sub-block from empty to full.  The Write Client can
write data to the ring buffer before flagging the start of the
observation.  In this way, it can clock data without activating the
Read Clients, and may change the state of the validity flags before
raising the start-of-data flag.

After raising the start-of-data flag, the Write Client will not write
data to a sub-block until its state is empty; after filling a
sub-block, it will change its state to full and set the data validity
byte offsets.  If the Write Client cannot obtain an empty sub-block,
an overflow condition will occur; this condition will be handled
according to the mode of operation:

\begin{itemize}

\item {\bf contiguous}, a contiguous stream of data is required (as is
  often the case in search observations).  In this case, data overflow
  is treated as an error that is propagated to the Command and Control
  software, and data acquisition is stopped.

\item {\bf discontinuous}, an end-of-data is written to the Data
  Block, and the Write Client continues to receive data from its
  source.  When the ring buffer has been cleared by the Read Client,
  the Write Client starts a new observation and data acquisition
  continues.

\item {\bf tolerant}, the Write Client will wait indefinitely for
  the next empty sub-block.

\end{itemize}

Other specifics of overflow handling will depend upon the application.
For example, discontinuous overflow handling may include a sending a
signal to to the Primary Node that instructs it to move on to the next
Secondary node.

\subsection{Read Clients}

One or more Read Clients may attach to the Data Block and read the
data from sub-blocks marked as full.  Only the bytes designated as
valid will be used from each sub-block.  Only a single, high-priority
Read Client will be given permission to change the state of a
sub-block from full to empty.  Read Clients will access sub-blocks in
contiguous order after the start-of-data flag is raised and until the
end-of-data condition is encountered.

\subsubsection{Example}

Consider a tight schedule, in which the time required to synchronize
and start data acquisition is considered too costly.  In this case,
some time may be saved by starting data acquisition before the
telescope is on source and continuing to acquire data while the
telescope slews between sources.

In this case, the Write Client will begin writing data to the Data
Block before raising the start-of-data flag.  The Write Client knows
the UTC time at which it started clocking data into the Data Block.
When the signal is given to the Write Client that the data became
valid at a certain UTC time, the Write Client can go back to that time
in its buffer, flag all data from that point to the present buffer as
valid, and raise the start-of-data flag.  In this way, the Write
Client can retroactively flag data as valid {\bf before the start of
the observation}.

In order to slew to the next source, a data invalid message must be
sent to the Write Client before the data becomes invalid.  The Write
Client will continue to clock data into the Data Buffer, but the data
will be marked as invalid and therefore will be ignored by the Read
Clients; the designated Read Client will simply mark the buffers as
empty as they are encountered. (This is why retroactive validity
flagging can be done only before the start of the observation.)

\section{Data Flow Write Clients}

Write Client software will read data from a device and write it to
the Data Block.

\subsection{DMA Client: {\tt dma2db}}

The DMA Client software, {\tt dma2db}, is responsible for transferring
data from the telescope to the Data Block.  It will talk directly to
the PiC through its PCI interface, start and stop the data transfer,
and record the UTC start time of the observation.  Data from the PiC
will be transferred to Primary node RAM via a Direct Memory Access
(DMA) card that is commercially available from Engineering Design Team
(EDT).  The DMA Client software will:

\begin{enumerate}

\item allocate a number of fixed memory buffers of a size and number
to be determined during the testing stage;

\item send start and stop instructions to the PiC via the PCI/DMA interface;

\item determine the UTC start time of the first sample recorded

\item copy filled DMA buffers to the Data Block; and

\item monitor the number of DMA buffers filled and copied, ensuring that
no data overflow occurs.

\end{enumerate}

\noindent
The DMA buffers will be separate from the Data Block buffers and
accessed only by the DMA card driver and {\tt dma2db}.  Once started,
DMA transfer will continue uninterrupted until a stop flag is raised
or an overflow occurs.

\subsection{Network Interface Client: {\tt nic2db}}

The software for network I/O will run on both Primary and Secondary
nodes.  The Data Flow Control software running on the Secondary nodes,
{\tt nic2db}, will open a port and listen for incoming Data Flow
Control connections from {\tt db2nic}, which will run on the Primary
nodes.  A single incoming channel will be connected and used to
establish a high-bandwidth data communication channel between a single
Primary node and a single Secondary node.  The protocol for the
network communications will be a simple, custom-built design on top of
internet sockets.  This may change in the future to some sort of
grid-based protocol.  Data received via this communication channel
will be copied to the Data Block in contiguous order.  Each packet of
data will be preceded by a copy of the Data Block header from the
Primary node.  This header will be copied to the Secondary node Data
Block.

The {\tt nic2db} software has the responsibility to monitor the Data
Block and ensure that there is sufficient space to hold incoming data
packets.  It will send a message to the Primary node if there is
insufficient space, and the Primary node will cease data transfer,
possibly initiating data transfer to the next in Secondary node in the
ring.

\section{Data Flow Read Clients}

Read Client software will read data from the Data Block and write it
to a device.

\subsection{Data Storage Client: {\tt db2disk}}

Writes data blocks to disk, breaking up data into files of arbitrary
length.  Each file will be preceded by the header block from the Data
Block.  Runs on either Primary or Secondary nodes, depending on the
mode of operation.  After each file is written to disk, an entry will
be added to an ASCII text log file; each entry will describe:
\begin{itemize}
\item the full path to the file
\item the time it was written
\item the size of the file
\item the time required to write the file
\item the observation identifier
\end{itemize}
This log file will be polled by the Configuration and Scheduling
software, which will add the information to a centralized database.

\subsection{Network Interface Client: {\tt db2nic}}

This software runs on the Primary nodes; it reads from the Data Block
and writes to one or more Secondary nodes, breaking up data into
packets of arbitrary length.  The total length of data sent to an
individual Secondary node will be independent of the Data Block buffer
sizes, and may depend on the overlap specified by the Configuration
and Scheduling software.  Header information (including all available
observation information as well as offset byte counts) will be sent
with each block of data transmitted to the Secondary nodes.

\chapter{Data Reduction Software}

\section{Operational Phases}

The data reduction software supports three models:

\begin{itemize}
\item {\bf offline} processing: performed after the recording has finished
\item {\bf simultaneous} processing: performed during the recording
\item {\bf diskless} processing: as above, but completely in memory
\end{itemize}

\subsection{Offline Data Reduction}

The data reduction software operates only on data files stored on
local disk space, and only after the recording has completed.  After
the observations have been completed, the data files from each Primary
node will be farmed out to the Secondary nodes.  After each file has
been copied, an entry will be added to the log file that is monitored
by the Configuration and Scheduling software, which will attend to the
data reduction as described in Chapter 6.

\subsection{Simultaneous Data Reduction}

The data reduction software runs during the observation.  In this
mode, data are sent directly to Secondary nodes and written to disk.
As before, an entry will be made in a log file and the Configuration
and Scheduling software will coordinate the data reduction.

\subsection{Diskless Data Reduction}

The data reduction software is able to keep up completely with the
flow of data on the Primary nodes.  In this case, the data is never
written to disk, and {\tt dspsr} operates as a Read Client directly
connected to the Data Block.  Another Read Client may be written that
will farm data reduction tasks to a grid computing facility.  In this
case, the data reduction Read Client will have to incorporate the
Configuration and Scheduling tasks.


\chapter{Command and Monitoring Software}

\section{Command}

Each element of the DADA software must be coordinated to operate as
a single instrument.  Therefore, many of the processes described in
the previous chapter will have to be synchronized and configured
through communications channels.  Some degree of synchronization will
be achieved through the hand-shaking protocol of the Data Block
specification.  Other communication requirements will be met through
internet socket connections.

\subsection{Data Block Communications}

Many of the Data Clients can be implemented as a persistent process,
like a daemon, that is configured once during an initialization stage
and runs automatically from that point onward.  Apart from
configuration, the behaviour of these automatic processes will depend
completely upon the state of the Data Block.

Two Read Client programs that can run in this manner are {\tt dbdisk}
and {\tt dbnic}.  These automatic processes need only start reading
from the Data Block when it is active, as determined by the behaviour
of the Write Client.  They start when the header is initialized and
stop when the end of data flag is raised.
Also, the {\tt nicdb} Write Client can be run as an automatic process
that starts when it receives packets from the Primary Node and ends
when the end of data message is received.
If it is shown that the operation of these Data Clients may have to
change from observation to observation, then there are two possible
solutions:
\begin{itemize}
\item Stop and restart the daemons with different configuration parameters
\item Enable socket communications that set configuration parameters
\end{itemize}

\subsection{Internet Socket Communications}

Certain processes will require internet socket communications in order
to be configured between observations and to start and stop the
observation.  In order that communications may be sent and received
during normal operation, the processes that require socket
communications will be multi-threaded.  The communication threads may
have lower priority than the main thread, if required.  More than one
communication channel may be opened to each process; however, only one
channel will be able to issue control commands.  The others may only
inquire about the status of the process.

All communications will be human readable, ASCII text.  This enables
interface testing using standard tools like telnet.  If large amounts
of binary data must be sent between the Control Interface and Data
Flow Control software, then it should be done using a separate
communication channel designated for this purpose.  Text commands will
be sent on a single line of text.  After every command received, the
process will respond with {\tt ok} or {\tt fail}, followed by any
additional information, and ending with the command prompt.

\subsubsection{DMA Data Client}

The {\tt dmadb} processes running on each of the primary nodes
require careful synchronization with the Control Interface software,
especially if they are all to be started on the same UTC second.  For
this reason, it will not suffice to remotely start the processes at
the desired beginning of each observation.  The processes must be
persistent and must maintain socket communications with the Control
Interface.  The Command Interface to {\tt dmadb} is described in
Section~\ref{sec:pwc}.

\section{Monitoring}

Monitoring processes will be run on all nodes in the instrument,
reporting on remaining disk space, CPU load, network traffic, etc.  At
the time of this writing, it is not clear if standard cluster
monitoring tools like Ganglia will suffice.  For example, it may prove
useful to have a regular report on which Secondary nodes are currently
receiving data.  This information would have to come from {\tt dbnic},
possibly via a socket connection to this process.

In addition to live monitoring, it may also prove useful to maintain a
database of relevant statistics, such as the average time required to
write a block of data to file or over the network.  These monitoring
tasks would be performed by the relevant process, {\tt dbdisk} and/or
{\tt dbnic} and communicated to a central database via some protocol.

\newpage
\section{Primary Write Client Command Interface}
\label{sec:pwc}

This section describes the behaviour of the Write Client software that
will run on each of the Primary nodes, known as the Primary Write
Client (PWC) software.  In the case of PuMa-II, the PWC is 
{\tt puma2\_dmadb}.

\subsection{Operational States}

The PWC has four main states of operation:
\vspace{-2mm}
\begin{itemize}
\item {\bf idle} waiting for configuration parameters
\vspace{-2mm}
\item {\bf prepared} configuration parameters received; waiting for start
\vspace{-2mm}
\item {\bf recording invalid} recording data in over-write mode
\vspace{-2mm}
\item {\bf recording valid} recording data in lock-step mode
\end{itemize}

\subsubsection{Idle State}

In the idle state, the PWC sleeps until configuration parameters are
sent from the control software.  All configuration paramters are sent
in a single ASCII header.  This header is copied to the Header Block,
and the PWC enters the {\bf prepared} state.

\subsubsection{Prepared State}

In the prepared state, the PWC sleeps until a start command is sent
from the control software.  There are three different start commands
that can be received in this state:

\begin{itemize}
\item {\tt\bf INV\_START} enter the {\bf recording invalid} state
\vspace{-2mm}
\item {\tt\bf START} enter the {\bf recording valid} state
\vspace{-2mm}
\item {\tt\bf START $\langle duration\rangle$} same as {\tt START}, record
	for the duraction specified in either {\it seconds}, {\it samples},
	or {\it HH:MM:SS}.
\end{itemize}
For each of the above commands, the PWC will enter the specified state
at the next available opportunity (for PuMa-II, on the next {\tt
SYSTICK}).

\subsubsection{Recording Invalid State}

In this state, the PWC software clocks data into the Data Block but
does not flag the data as valid.  The PWC will overwrite the data in
each sub-block, and will remain in this state until one of the
following commands is received:

\begin{itemize}
\item {\tt\bf STOP} enter the {\bf idle} state immediately
\vspace{-2mm}
\item {\tt\bf VAL\_START YYYY-MM-DD-hh:mm:ss} raise the valid data flag
	at the specified UTC time in the data stream and enter the {\bf
	recording valid} state
\vspace{-2mm}
\item {\tt\bf VAL\_START YYYY-MM-DD-hh:mm:ss $\langle duration\rangle$} same 
	as {\tt VAL\_START}, record for the duraction specified in either
	{\it seconds}, {\it samples}, or {\it HH:MM:SS}.
\end{itemize}

Note that the UTC time specified in the first argument to {\tt
VAL\_START} may be any time in the future.  If it is in the past, then
the difference between the specified UTC and the present cannot be
greater than the amount of time corresponding to the length of the
Data Block.

\subsubsection{Recording Valid State}

In this state, the PWC software clocks data into the Data Block, flags
the data as valid, and will not overwrite a sub-block until it has
been flagged as cleared.  The PWC will remain in this state until one
of the following commands is received:

\begin{itemize}
\item {\tt\bf STOP} enter the {\bf idle} state immediately
\vspace{-2mm}
\item {\tt\bf STOP YYYY-MM-DD-hh:mm:ss} enter the {\bf idle} state
	at the specified time
\vspace{-2mm}
\item {\tt\bf VAL\_STOP YYYY-MM-DD-hh:mm:ss} raise the end of data flag
	at the specified UTC time in the data stream and enter the {\bf
	recording invalid} state
\end{itemize}

Note that the UTC time specified in the first argument to both {\tt STOP}
and {\tt VAL\_STOP} {\bf must} be in the future.



\chapter{Control Interface Software}


\chapter{Configuration and Scheduling Software}

Depending upon the objectives of the experiment, DADA configuration
may change from observation to observation.

\section{Instrumental Configuration}

Based upon the parameters of observation, the {\tt dada} control
program will set up various operational parameters, including:
\begin{itemize}
\item buffer sizes: DMA, Data Block, file and network I/O
\item number of Primary nodes
\item number of Secondary nodes
\item assignment of Secondary to Primary nodes
\item operational mode: offline, simultaneous, or diskless
\item overlap required between Secondary nodes
\end{itemize}
These operational parameters will be stored in a {\bf configuration
database}, which will specify the instrument configuration and data
reduction requirements for various combinations of source, receiver,
centre frequency, band width, etc.

Entries in the configuration database may have an expiration date
associated with them.  In this manner, the observer may specify
special configuration and/or reduction options for a specific
experiment without permanently changing the default behaviour.

\section{Data Reduction}

The Configuration and Scheduling program, {\tt dadaskd}, will be used
to configure and schedule all data reduction operations.  Before the
{\bf diskless} mode of data reduction is implemented, all data will
exist as a file on either the Primary or Secondary nodes.  Whenever a
file is written to disk, an entry will be registered in a centralized
{\bf observations database}, which will contain basic header
information such as
\begin{itemize}
\item source name

\item start time (UTC)

\item centre frequency (MHz)

\item band width (MHz)
\end{itemize}
as well as the location (machine and file name) of the data.  Each
entry will also contain a time-stamped list of {\bf performed
operations}, describing when the data was written, when and how it was
processed, when it was deleted, etc.  The header information will be
used to find matches in the configuration database.  An observation
may be processed in multiple ways, as specified by the list of {\bf
requested operations} in the configuration database entry.

If it is possible to achieve two requested operations in one execution
of {\tt dspsr} then this will be done.  Otherwise, data reduction
operations will be performed one at a time.  After each operation is
completed, it will be recorded in the list of performed operations of
the observations database entry.

The scheduling software will periodically check or poll the
observations database.  Any entries that require data reduction will
be scheduled according to the data reduction parameters of the
requested operations in the corresponding configuration database
entry.  An observation will be considered completely processed when
the list of performed operations is equal to the list of requested
operations.  At this point, the raw data will be deleted or archived.

\chapter{APSR Objectives}
\label{app:dada_todo}

APSR will depend upon the following new techniques, which must be
implemented, demonstrated, and rigorously tested.

\begin{itemize}

\item streaming UDP packets from board to nodes

\item optimal use of multiple cores

\item processing directly from RAM 

\item simultaneously processing and writing to disk

\item processing data written to disk \\

\end{itemize}

\noindent
In addition, the following management software must be written/adapted:

\begin{itemize}

\item User Interfaces: text, gui, web-based?

\item TCS Interface

\item Collection, presentation, and archival of reduced data

\end{itemize}
\section{Streaming UDP packets from board to nodes}

Can it be done while processing? How should dropped packets be handled? \\
{\bf Who:} Grant and Willem \\
{\bf Status:} hardware ready, no software

\section{Optimal use of multiple cores}

Is a multi-threaded solution the most efficient? \\
{\bf Who:} Willem \\
{\bf Status:} implemented but not optimized, currently debugging

\section{Processing directly from RAM}

Should multiple threads be triggered to return results before reaching
end-of-data, or should end-of-data be induced at regular intervals? \\
{\bf Who:} Willem \\
{\bf Status:} implemented, tested on single archive only

\section{Simultaneously processing and writing to disk}

Should a monitor be used to schedule the two tasks intelligently, or
should priorities be assigned to the processes? \\
{\bf Who:} Willem {\it and Andrew?} \\
{\bf Status:} requires redesign of ring buffer

\section{Processing data written to disk}

Can the redistribution and processing of data written to disk take
place during easy observations, or only offline?  How should different
data sets be scheduled?  \\
{\bf Who:} Willem {\it and Andrew?} \\
{\bf Status:} simple plan: process easiest (lowest DM) observations first



\chapter{Summary}

The following table summarizes the software that will be developed,
its basic functionality, and the machine on which it will run.

\vspace{5mm}

\begin{tabular}{l|p{8cm}|l}

Name & \multicolumn{1}{c}{Function} & Machine \\ \hline

{\tt dma2db} & Transfers data from EDT buffers to the Data Block.
	& Primary \\

{\tt db2disk} & Reads data from the Data Block and writes it, with
	header information, to disk. & Both \\

{\tt db2nic} & Reads data from the Data Block and sends it, with
	header information, to a Secondary node. & Primary \\

{\tt nic2db} & Receives data from Primary node and writes it to the 
	Data Block. & Secondary \\

{\tt dspsr} & Attaches to the Data Block and processes raw data
	according to specification, writing results to disk. &
	Secondary \\

{\tt puma2} & Connects to the Command and Control interface of the
	various Primary nodes.  Accepts and maintains external
	text-based control connection.  & fixed \\

{\tt puma2tms} & Translates TMS control structure packets into
	text-based commands as accepted by {\tt puma2} control
	connection. & fixed \\

{\tt puma2tui} & Textual user interface connects to {\tt puma2},
	displays various quantities of interest, and controls
	the instrument. & variable \\

{\tt puma2skd} & High-level code that schedules and records
        data reduction operations & fixed \\

\end{tabular}



\appendix
\chapter{Data Block Ring Buffer}

In this chapter, the Data Block API is specified in detail.  The Data
Block is the ring buffer through which the primary data flow will take
place on both primary and secondary nodes in the PuMa-II cluster.
Access to the ring buffer shared memory is controlled by an
inter-process communication semaphore.

The Data Block API includes software for creating and initializing the
shared memory and semaphore resources, locking the shared memory into
physical RAM, connecting to the ring buffer, writing data to the ring
buffer and reading data from the ring buffer.

\section{Creation, Connection, and Destruction}

The Data Block ring buffer is accessed through a data type named {\tt
ipcbuf\_t}, which is declared and initialized as in the following
example:
\begin{verbatim}
#include "ipcbuf.h"
ipcbuf_t ringbuf = IPCBUF_INIT;
\end{verbatim}
To create a ring buffer, call
\begin{verbatim}
int ipcbuf_create (ipcbuf_t* ptr, int key, uint64 nbufs, uint64 bufsz);
\end{verbatim}
\vspace{-3mm}
\begin{itemize}
\item {\tt ptr} is a pointer to an unallocated ring buffer handle
\item {\tt key} is a unique identifier (range of acceptable values???)
\item {\tt nbufs} is the number of sub-blocks in the ring buffer
\item {\tt bufsz} is the size of each sub-block in the ring buffer
\end{itemize}
After the ring buffer has been created, it is ready for use.   The ring
buffer resources will remain available until calling
\begin{verbatim}
int ipcbuf_destroy (ipcbuf_t* ptr);
\end{verbatim}
\vspace{-3mm}
\begin{itemize}
\item {\tt ptr} is a pointer to an allocated ring buffer handle
\end{itemize}
That is, even if the process that created the ring buffer exits, the
shared memory and semaphore resources will remain allocated in
computer memory.  In order to connect to a previously created Data
Block ring buffer, call
\begin{verbatim}
int ipcbuf_connect (ipcbuf_t* ptr, int key);
\end{verbatim}
\vspace{-3mm}
\begin{itemize}
\item {\tt ptr} is a pointer to an unallocated ring buffer handle
\item {\tt key} is the unique identifier passed to {\tt ipcbuf\_create}
\end{itemize}
To disconnect, call
\begin{verbatim}
int ipcbuf_disconnect (ipcbuf_t* ptr);
\end{verbatim}
\vspace{-3mm}
\begin{itemize}
\item {\tt ptr} is a pointer to a connected ring buffer handle
\end{itemize}
Note that, after calling {\tt ipcbuf\_create}, the process is connected
to the newly-created ring buffer and it is not necessary to call 
{\tt ipcbuf\_connect}.  Similarly, after calling {\tt ipcbuf\_destroy},
it is not necessary (or possible) to call {\tt ipcbuf\_disconnect}.
After the process is connected to the Data Block ring buffer, it is
possible to write or read data.

\subsection{Locking into Physical RAM}

In order to ensure that the Data Block ring buffer remains in RAM and
is not swapped out by the virtual memory manager, call
\begin{verbatim}
int ipcbuf_lock_shm (ipcbuf_t* ptr);
\end{verbatim}
and, to unlock,
\begin{verbatim}
int ipcbuf_unlock_shm (ipcbuf_t* ptr);
\end{verbatim}


\section{Writing to the Data Block}

After connecting to the Data Block ring buffer, it is
possible to write data to it.

\subsection{Locking and Unlocking Write Access}

Naturally, only one process may write data to the ring buffer;
therefore, the Write Client must first lock write access to the buffer
by calling,
\begin{verbatim}
int ipcbuf_lock_write (ipcbuf_t* ptr);
\end{verbatim}
\vspace{-3mm}
\begin{itemize}
\item {\tt ptr} is a pointer to a connected ring buffer handle.
\end{itemize}
Similarly, write permission may be relinquished by calling
\begin{verbatim}
int ipcbuf_unlock_write (ipcbuf_t* ptr);
\end{verbatim}
\vspace{-3mm}
\begin{itemize}
\item {\tt ptr} is a pointer to a connected ring buffer handle.
\end{itemize}

\subsection{Write Loop}

After locking write access to the Data Block ring buffer, the Write
Client will generally enter a loop in which it
\begin{enumerate}
\item requests the next sub-block to which data may be written, 
\item fills the sub-block
\item marks the sub-block as filled
\end{enumerate}
Step 1 is performed by calling
\begin{verbatim}
char* ipcbuf_get_next_write (ipcbuf_t* ptr);
\end{verbatim}
\vspace{-3mm}
\begin{itemize}
\item {\tt ptr} is a pointer to a connected ring buffer handle
\item RETURN value is the pointer to the next available sub-block
\end{itemize}
\begin{verbatim}
int ipcbuf_mark_filled (ipcbuf_t* ptr, uint64 nbytes);
\end{verbatim}
\vspace{-3mm}
\begin{itemize}
\item {\tt ptr} is a pointer to a connected ring buffer handle
\item {\tt nbytes} is the number of valid bytes in the sub-block
\end{itemize}

If {\tt nbytes} is less than the number of bytes in each sub-block, as
set by the {\tt bufsz} argument to {\tt ipcbuf\_create}, then an
end-of-data condition is set.

\subsection{Writing before Start-of-Data}
\label{sec:prewrite}

By default, when a Data Block ring buffer is created, the
start-of-data state is enabled and any data written by the Write
Client will be made available to the Read Client.  However, in some
cases it may be useful for the Write Client to write data to the Data
Block before making it available to the Write Client.  For example,
the trigger to begin data acquisition may arrive later than the
desired data acquisition start time.

To begin writing data before the actual start of valid data, it is
necessary to first disable the start-of-data flag by calling
\begin{verbatim}
int ipcbuf_disable_sod (ipcbuf_t* ptr);
\end{verbatim}
\vspace{-3mm}
\begin{itemize}
\item {\tt ptr} is a pointer to a connected ring buffer handle
\end{itemize}
The Write Client may then enter a loop identical to that described in
the previous section: requesting, filling, and marking.  However, when
the start-of-data flag is disabled, the message that a sub-block has
been filled is not passed on to the Read Client and the Write Client
will over-write filled sub-blocks as necessary.  The Write Client
raises the start-of-data flag by calling
\begin{verbatim}
int ipcbuf_enable_sod (ipcbuf_t* ptr, uint64 st_buf, uint64 st_byte);
\end{verbatim}
\vspace{-3mm}
\begin{itemize}
\item {\tt ptr} is a pointer to a connected ring buffer handle
\item {\tt st\_buf} is the absolute count of the first valid sub-block
\item {\tt st\_byte} is the first valid byte in the first valid sub-block
\end{itemize}

Note that {\tt st\_buf} is an absolute sub-block count, equal to the
total number of sub-blocks filled before the first valid sub-block.
Naturally, it is not possible to raise the start-of-data flag for a
buffer that has already been over-written.  Therefore, the start
sub-block count plus the total number of sub-blocks must be greater
than the current sub-block count, or
\begin{verbatim}
st_buf > ipcbuf_get_write_count - ipcbuf_get_nbufs
\end{verbatim}


\section{Reading from the Data Block}

After connecting to the Data Block ring buffer, it is possible to read
data from it.  

\subsection{Locking and Unlocking Read Access}

Only one process may remove data from the ring buffer by flagging it
as cleared.  This process, the Read Client, must first lock read
access to the buffer by calling,
\begin{verbatim}
int ipcbuf_lock_read (ipcbuf_t* ptr);
\end{verbatim}
\vspace{-3mm}
\begin{itemize}
\item {\tt ptr} is a pointer to a connected ring buffer handle
\end{itemize}
Similarly, read permission may be relinquished by calling
\begin{verbatim}
int ipcbuf_unlock_read (ipcbuf_t* ptr);
\end{verbatim}
\vspace{-3mm}
\begin{itemize}
\item {\tt ptr} is a pointer to a connected ring buffer handle
\end{itemize}

\subsection{Read Loop}

After locking read access to the Data Block ring buffer, the Read Client
will generally enter a loop in which it
\begin{enumerate}
\item requests the next sub-block containing data, 
\item operates on the data in the sub-block
\item marks the sub-block as cleared
\end{enumerate}
Step 1 is performed by calling
\begin{verbatim}
char* ipcbuf_get_next_read (ipcbuf_t* ptr, uint64* bytes);
\end{verbatim}
\vspace{-3mm}
\begin{itemize}
\item {\tt ptr} is a pointer to a connected ring buffer handle
\item {\tt bytes} will be set to the number of valid bytes in the sub-block
\item RETURN value is the pointer to the first valid byte in the sub-block
\end{itemize}
Step 3 is performed by calling
\begin{verbatim}
int ipcbuf_mark_cleared (ipcbuf_t* ptr);
\end{verbatim}
\vspace{-3mm}
\begin{itemize}
\item {\tt ptr} is a pointer to a connected ring buffer handle
\end{itemize}


\section{Data Block Abstraction}

The fact that data is written to a ring buffer with individual
sub-blocks may be abstracted from high-level code, thereby allowing
the Data Block to be treated like any other storage device.  This
abstraction is accessed through a data type named {\tt
ipcio\_t}, which is declared and initialized as in the following
example:
\begin{verbatim}
#include "ipcio.h"
ipcio_t data_block = IPCIO_INIT;
\end{verbatim}
To create an abstract ring buffer, call
\begin{verbatim}
int ipcio_create (ipcio_t* ptr, int key, uint64 nbufs, uint64 bufsz);
\end{verbatim}
\vspace{-3mm}
\begin{itemize}
\item {\tt ptr} is a pointer to an unallocated abstract ring buffer handle
\item {\tt key} is a unique identifier (range of acceptable values???)
\item {\tt nbufs} is the number of sub-blocks in the ring buffer
\item {\tt bufsz} is the size of each sub-block in the ring buffer
\end{itemize}
After the abstract ring buffer has been created, it is ready for use
and the resources will remain available until calling
\begin{verbatim}
int ipcio_destroy (ipcio_t* ptr);
\end{verbatim}
\vspace{-3mm}
\begin{itemize}
\item {\tt ptr} is a pointer to an allocated abstract ring buffer handle
\end{itemize}
In order to connect to a previously created Data Block ring buffer,
call
\begin{verbatim}
int ipcio_connect (ipcio_t* ptr, int key);
\end{verbatim}
\vspace{-3mm}
\begin{itemize}
\item {\tt ptr} is a pointer to an unallocated abstract ring buffer handle
\item {\tt key} is the unique identifier passed to {\tt ipcio\_create}
\end{itemize}
To disconnect, call
\begin{verbatim}
int ipcio_disconnect (ipcio_t* ptr);
\end{verbatim}
\vspace{-3mm}
\begin{itemize}
\item {\tt ptr} is a pointer to a connected ring buffer handle
\end{itemize}
Note that, after calling {\tt ipcio\_create}, the process is connected
to the newly-created ring buffer and it is not necessary to call 
{\tt ipcio\_connect}.  Similarly, after calling {\tt ipcio\_destroy},
it is not necessary (or possible) to call {\tt ipcio\_disconnect}.
After the process is connected to the Data Block ring buffer, it is
possible to write or read data.

\subsection{Read/Write Access}

To begin writing or reading data through the abstract ring buffer interface,
the Write or Read Client must first call
\begin{verbatim}
int ipcio_open (ipcio_t* ptr, char rw);
\end{verbatim}
\vspace{-3mm}
\begin{itemize}
\item {\tt ptr} is a pointer to a connected abstract ring buffer handle.
\item {\tt rw} is either `W', `w', `R', or `r'
\end{itemize}
The meanings of the {\tt rw} character codes are as follows:
\begin{itemize}
\item {\bf W} open for writing valid data
\item {\bf w} open for writing before valid data
\item {\bf R} open for reading as primary Read Client
\item {\bf r} open for reading as secondary Read Client
\end{itemize}
Similarly, write or read permission may be relinquished by calling
\begin{verbatim}
int ipcio_close (ipcio_t* ptr);
\end{verbatim}
\vspace{-3mm}
\begin{itemize}
\item {\tt ptr} is a pointer to a connected ring buffer handle.
\end{itemize}
When the Write Client calls {\tt ipcio\_close}, an end-of-data is written
to the Data Block. Data are written to the abstract ring buffer by calling
\begin{verbatim}
ssize_t ipcio_write (ipcio_t* ptr, char* buf, size_t nbytes);
\end{verbatim}
\vspace{-3mm}
\begin{itemize}
\item {\tt ptr} is a pointer to a connected abstract ring buffer handle
\item {\tt buf} is a pointer to the data to be written
\item {\tt nbytes} is the number of bytes to be written
\item RETURN value is the number of bytes written, or -1 on error
\end{itemize}
Data are read from the abstract ring buffer by calling
\begin{verbatim}
ssize_t ipcio_read (ipcio_t* ptr, char* buf, size_t nbytes);
\end{verbatim}
\vspace{-3mm}
\begin{itemize}
\item {\tt ptr} is a pointer to a connected abstract ring buffer handle
\item {\tt buf} is a pointer to the buffer to be filled
\item {\tt nbytes} is the number of bytes to be read
\item RETURN value is the number of bytes read, or -1 on error
\end{itemize}

\subsection{Inheritance in C}

The {\tt ipcio\_t} data type {\em is a} {\tt ipcbuf\_t}.  That is, the
pointer to the base address of an {\tt ipcio\_t} data structure can be
passed, after casting, to all of the functions that receive a pointer
to the base address of an {\tt ipcbuf\_t} data structure.  This
inheritance allows the {\tt ipcio\_t} data type to be treated as
though it were a {\tt ipcbuf\_t} data type. For example,
\begin{verbatim}
#include "ipcio.h"
ipcio_t data_block = IPCIO_INIT;
ipcio_connect (&data_block, 0xc2);
uint64 nbufs = ipcbuf_get_nbufs ((ipcbuf_t*)(&datablock));
\end{verbatim}

\subsection{Writing before Start-of-Data}

To begin writing to the Data Block before the actual start of valid data, 
it is necessary to call {\tt ipcio\_open} with the `w' argument.
The Write Client may then raise the start-of-data flag by calling
\begin{verbatim}
int ipcio_start (ipcio_t* ptr, uint64 st_byte);
\end{verbatim}
\vspace{-3mm}
\begin{itemize}
\item {\tt ptr} is a pointer to a connected ring buffer handle
\item {\tt offset} is the byte offset after {\tt ipcio\_open}
\end{itemize}
To write an end-of-data flag without closing
the abstract ring buffer, the Write Client must call
\begin{verbatim}
int ipcio_stop (ipcio_t* ptr);
\end{verbatim}
\vspace{-3mm}
\begin{itemize}
\item {\tt ptr} is a pointer to a connected abstract ring buffer handle
\end{itemize}
The end-of-data flag is raised after the last byte written.  
As long as each call to {\tt ipcio\_start} is matched by a corresponding
call to {\tt ipcio\_stop}, these functions can be called an arbitrary
number of times between the call to {\tt ipcio\_open} 
and {\tt ipcio\_close}.


\chapter{Header Block}

In this chapter, the Header Block API is specified in detail.  The
Header Block is the buffer through which the Write Client communicates
auxilliary information about the data stream to the Read Client(s).
Access to the Header Block shared memory is controlled through the
same semaphore-based interface as the Data Block.  The major difference
is that the Header Block ring buffer is opened only once, and each
sub-block written to the ring buffer represents a unique header.

The Data Block API includes software for creating and initializing the
shared memory and semaphore resources, locking the shared memory into
physical RAM, connecting to the ring buffer, writing data to the ring
buffer and reading data from the ring buffer.

\section{Setting and Getting Header Values}

Each sub-block of the Header Block ring buffer contains an ASCII text
description of the data that is currently being written to the Data
Block.  An ASCII representation offers a number of significant
advantages; it is human readable, and there is no need to worry about
byte alignment and endian issues.  Data attributes are stored as
keyword-value pairs, each separated by a new line.

Given the base address of an ASCII header block, keyword-value pairs
may be conveniently set and queried using the {\tt ascii\_header} API,
which includes:
\begin{verbatim}
int ascii_header_set (char* header, const char* keyword,
                      const char* format, ...);
\end{verbatim}

\begin{itemize}
\item {\tt header} pointer to a null-terminated ASCII header block

\item {\tt keyword} the name of the attribute to be set

\item {\tt format} an fprintf-style formatting string

\item {\tt ...} the value(s) to be printed following the keyword
\end{itemize}
and
\begin{verbatim}
int ascii_header_get (const char* header, const char* keyword,
                      const char* format, ...);
\end{verbatim}

\begin{itemize}
\item {\tt header} pointer to a null-terminated ASCII header block

\item {\tt keyword} the name of the attribute to be queried

\item {\tt format} an fscanf-style formatting string

\item {\tt ...} the value(s) to be parsed following the keyword
\end{itemize}

\section{Examples}

\begin{verbatim}
#include "ascii_header.h"

[...]

  char ascii_header[MY_ASCII_HEADER_SIZE] = "";

  char* telescope_name = "westerbork";
  ascii_header_set (ascii_header, "TELESCOPE", "%s", telescope_name);

  float bandwidth = 20.0;
  ascii_header_set (ascii_header, "BW", "%f", float);

[...]

  double centre_frequency;
  ascii_header_get (ascii_header, "FREQ", "%lf", &centre_frequency);

  float min, max;
  ascii_header_get (ascii_header, "GAIN", "%f %f", &min, &max);
\end{verbatim}

Note that the formatting of a ``value'' is completely flexible; any
number of values can be associated with a single ASCII header keyword.
If the keyword does not exist when {\tt ascii\_header\_set} is called,
it will be added; if it does exists, its value will be replaced with
the new value.

It is also important to note that in the current API, {\tt
ascii\_header\_set} has no idea about the size of the ASCII header.
Therefore, it is up to the user to ensure that new attributes will not
cause the length of the string to exceed that of the allocated memory.

\section{PuMa-II header parameters}

The following header parameters will be included in the PuMa-II ASCII header:
\begin{itemize}
\item{\tt HEADER} name of the header

\item{\tt HDR\_VERSION} version of the header

\item{\tt HDR\_SIZE} size of the header in bytes

\item{\tt PRIMARY} host name of the Primary Node on which the data was acquired

\item{\tt HOSTNAME} host name of the machine on which data were written

\item{\tt FILE\_NAME} full path of the file to which data were written

\item{\tt FILE\_SIZE} requested size of data files

\item{\tt FILE\_NUMBER} number of data files written prior to this one

\item{\tt OBS\_ID} the identifier for the observations

\item{\tt UTC\_START} the UTC of the first sample in the observation

\item{\tt MJD\_START} the MJD of the first sample

\item{\tt OBS\_OFFSET} the number of bytes from the start of the observation

\item{\tt OBS\_OVERLAP} the amount by which neighbouring files overlap
\end{itemize}

\subsection{Notes}

The {\tt HDR\_VERSION} parameter should monotonically increase, and
change only when a fundamental change in the interpretation of the
header attributes is required.

The {\tt OBS\_ID} parameter must uniquely identify the data stream.
As multiple data streams will be acquired simultaneously, the unique
identifier will describe both the epoch/sequence number and the data
stream/band.

The {\tt OBS\_OFFSET} parameter is added only after the data leaves
the Data Block of the Primary Node.  It counts the number of bytes
offset from the start of the observation of the first byte of data
described by the header.  For instance, if the Read Client is writing
files to disk, it would place in the ASCII header of each file written
to disk an {\tt OBS\_OFFSET} equal to the number of bytes that
preceded the first byte of data in the file.  If the Read Client is
writing data to a Secondary Node via a network interface, it would set
{\tt OBS\_OFFSET} to the number of bytes that preceded the first byte
of data to follow the header in the transmission.

The {\tt OBS\_OVERLAP} parameter is used by {\tt db2nic} to determine
the amount by which neighbouring data segments that are sent to
different Secondary nodes should overlap.



\chapter{Operational Logging}

The various messages produced by the data acquisition software must be
logged and/or communicated to possibly more than one listener.  Therefore,
all messages will be sent using the {\tt multilog} API.  A multilog session
is opened by calling
\begin{verbatim}
multilog_t* multilog_open (const char* program_name, char syslog);
\end{verbatim}

\begin{itemize}
\item {\tt program\_name} the name of the program will precede each log message

\item {\tt syslog} if non-zero, all messages are cc'd to syslog
\end{itemize}
and closed by calling
\begin{verbatim}
int multilog_close (multilog_t* log);
\end{verbatim}

\begin{itemize}
\item {\tt log} pointer to an open multilog session
\end{itemize}
Once opened, file streams may be added by calling
\begin{verbatim}
int multilog_add (multilog_t* log, FILE* fptr);
\end{verbatim}

\begin{itemize}
\item {\tt log} pointer to an open multilog session

\item {\tt fptr} pointer to an open file stream
\end{itemize}
Messages are written to all file streams (and syslog, if enabled) by calling
\begin{verbatim}
int multilog (multilog_t* log, int priority, const char* format, ...);
\end{verbatim}

\begin{itemize}
\item {\tt log} pointer to an open multilog session

\item {\tt priority} a {\tt syslog} priority

\item {\tt format} an fprintf-style formatting string

\item {\tt ...} the value(s) to be printed according to the format
\end{itemize}

Output messages are assigned a priority as described in the man page
for the standard C {\tt syslog} utility.

\section{Example}

\begin{verbatim}
#include "multilog.h"

[...]

  /* open a connection to syslogd using the standard C facility */
  openlog ("dada_db2disk", LOG_CONS, LOG_USER);

  /* open a multilog session that will use syslog */
  multilog_t* log = multilog_open (1);

  /* copy all messages to the standard error */
  multilog_add (log, stderr);

  /* write a message */
  char* world_name = "Earth";
  int world_number = 1;

  multilog (log, LOG_INFO, "Hello %s %d", world_name, world_number);
\end{verbatim}

\chapter{Socket Communications}

\section{Examples}
\subsection{Example Server}

\begin{verbatim}
#include "sock.h"

[...]

  char hostname [100];
  int port = 20013;

  /* Ask for the fully qualified hostname ... */
  sock_getname (hostname, 100, 1);
  /* ... or, ask for the IP address */
  sock_getname (hostname, 100, 0);

  int sfd = sock_create (&port);
  if (sfd < 0)
    perror ("Error creating socket");

  fprintf (stderr, "listening on %s %d\n", hostname, port);
  int cfd = sock_accept (sfd);
  if (cfd < 0)
    perror ("Error accepting connection");
\end{verbatim}
The open file descriptor returned by {\tt sock\_accept} may be used in
calls to the standard C I/O routines, {\tt read} and {\tt write} as
well as {\tt send} and {\tt recv}.  Furthermore, the open file
descriptor can be converted into a stream by calling {\tt fdopen}.  It
is important to note that sockets do not support seeking; therefore, a
stream should be opened for only read or write access, never both.
For example.
\begin{verbatim}
  /* two separate I/O streams are required for reading and writing */
  FILE* sockin = fdopen (cfd, "r");
  FILE* sockout = fdopen (cfd, "w");

  /* line buffer the socket stream output */
  setvbuf (sockout, 0, _IOLBF, 0);
\end{verbatim}
The server can now read from {\tt sockin} using standard C I/O stream
routines such as {\tt fscanf}, {\tt fread}, and {\tt fgets}.  It can
also write to {\tt sockout} using {\tt fprintf} and {\tt fwrite}.
The call to {\tt setvbuf} is important; without this call, it would
be necessary to {\tt fflush} the output stream to ensure that messages
are communicated immediately.

\subsection{Example Client}

\begin{verbatim}
#include "sock.h"

[...]

  char* hostname = "apsr0.atnf.csiro.au";
  int port = 20013;

  /* Connect to the specified host and port */
  int cfd = sock_open (hostname, port);
  if (cfd < 0)
    perror ("Error opening socket");

  fprintf (stderr, "connected to %s %d\n", hostname, port);
\end{verbatim}
As with the server, the socket file descriptor may be accessed using
standard C I/O stream routines by calling {\tt fdopen}.

\chapter{Primary Write Client Command Interface}
\label{app:dada_pwc}



\chapter{Primary Write Client Main Loop}
\label{app:dada_pwc_main}

In this chapter, the Primary Write Client Main Loop API is specified
in detail.  The Primary Write Client (PWC) Main Loop implements the
PWC Data Flow Control described in Chapter~\ref{sec:pwc_main}.  The
PWC Main Loop is implemented as a function, {\tt dada\_pwc\_main},
which receives a pointer to a struct, {\tt dada\_pwc\_main\_t}, as its
only argument.  The member variables of this struct must be properly
set up before the Main Loop is entered, as described in the following
section.

\section{Initialization}

The Primary Write Client Main Loop structure is created as follows:
\begin{verbatim}
#include "dada_pwc_main.h"
dada_pwc_main_t* pwcm = dada_pwc_main_create ();
\end{verbatim}
After creation, the following member variables must be initialized:
\begin{enumerate}

\item {\bf {\tt pwc}} a pointer to a {\tt dada\_pwc\_t} struct; 
   this struct is used to communicate between the thread that parses
   external commands from the Command Interface and the thread that
   runs the PWC Main Loop.  The {\tt pwc} member variable can simply
   be set as in the following example:
\begin{verbatim}
/* create the command communication structure */
pwcm->pwc = dada_pwc_create ();

/* start the control connection server */
if (dada_pwc_serve (pwcm->pwc) < 0)
  /* report an error message */
\end{verbatim}

\item {\bf {\tt data\_block}} pointer to the {\tt ipcio\_t} struct 
	that is connected to the Data Block

\item {\bf {\tt header\_block}} pointer to the {\tt ipcio\_t} struct
	that is connected to the Header Block

\item {\bf {\tt start\_function}} a pointer to a function that starts the 
  data transfer:
\begin{verbatim}
time_t start_function (dada_pwc_main_t*, time_t utc);
\end{verbatim}
  This function receives the UTC start time, {\tt utc} at which the
  data transfer should begin.  If {\tt utc} equals zero, the function
  should start the data transfer at the soonest available opportunity.
  This function should take care of everything required to start the
  data transfer, and should return the UTC of the first time sample to
  be transfered or zero on error.

\item {\bf {\tt buffer\_function}} a pointer to a function that returns
  the next buffer to be written to the Data Block:
\begin{verbatim}
void* buffer_function (dada_pwc_main_t*, uint64_t* size);
\end{verbatim}
  This function should return the base address of the next buffer or
  the {\tt NULL} pointer on error.  The size of the buffer (in bytes)
  should be returned in the {\tt size} argument.

\item {\bf {\tt stop\_function}} a pointer to a function that stops
  the data transfer:
\begin{verbatim}
int stop_function (dada_pwc_main_t*);
\end{verbatim}
  This function should perform any tasks required to stop the data
  transfer before returning to the idle state.  This function should return
  a value less than zero in the case of error.

\item {\bf {\tt context}} [optional] pointer to any additional information.
  Should the implementation of any of the above three functions
  require access to other information, a pointer to this information
  can be stored in the {\tt context} member variable and retrieved by
  casting this member inside the function. e.g.
\begin{verbatim}
  struct puma2_t {
    EdtDev* edt_p;
    pic_t pic;
    unsigned buf_size;
  };

  void* puma2_buffer_function (dada_pwc_main_t* pwcm, uint64_t* size)
  {
    struct puma2_t* xfer = (struct puma2_t*) pwcm->context;
    *size = xfer->buf_size;
    return edt_wait_for_buffers (xfer->edt_p, 1);
  }

  [...]

  struct puma2_t xfer_data;

  pwcm->context = &xfer_data;
  pwcm->buffer_function = puma2_buffer_function;

  [etc...]
\end{verbatim}

\end{enumerate}

\chapter{Primary Read Client Main Loop}
\label{app:dada_prc_main}

In this chapter, the Primary Read Client Main Loop API is specified in
detail.  The Primary Read Client (PRC) Main Loop implements the PRC
Data Flow Control.  The PRC Main Loop is implemented as a function,
{\tt dada\_prc\_main}, which receives a pointer to a struct, {\tt
dada\_prc\_main\_t}, as its only argument.  The member variables of
this struct must be properly set up before the Main Loop is entered,
as described in the following section.

\section{Initialization}

The Primary Read Client Main Loop structure is created as follows:
\begin{verbatim}
#include "dada_prc_main.h"
dada_prc_main_t* prcm = dada_prc_main_create ();
\end{verbatim}
After creation, the following member variables must be initialized:
\begin{enumerate}

\item {\bf {\tt log}} pointer to the {\tt multilog\_t} struct
	that will be used for status and error reporting

\item {\bf {\tt data\_block}} pointer to the {\tt ipcio\_t} struct 
	that is connected to the Data Block

\item {\bf {\tt header\_block}} pointer to the {\tt ipcio\_t} struct
	that is connected to the Header Block

\item {\bf {\tt open\_function}} pointer to the function that opens
	the file descriptor to which data will be written:
\begin{verbatim}
int open_function (dada_prc_main_t*);
\end{verbatim}
  This function should initialize the following member
  variables of {\tt dada\_prc\_main\_t}:
\begin{itemize}
\item {\bf {\tt fd}} file descriptor to which data will be written
\item {\bf {\tt transfer\_bytes}} total number of bytes to write
\item {\bf {\tt optimal\_bytes}} optimal number of bytes to write each time
\item {\bf {\tt header}} [optional] header to be written to the file descriptor
\end{itemize}
  The {\tt header} member variable will be initialized before calling
  {\tt open\_function} and will be written to the file descriptor
  after this function returns. This function should return a value
  less than zero in the case of any error.
\newpage
\item {\bf {\tt close\_function}} pointer to the function that closes
	the file descriptor after data has been written:
\begin{verbatim}
int close_function (dada_prc_main_t*, uint64_t bytes_written);
\end{verbatim}
  The {\tt bytes\_written} argument specifies the total number of bytes
  successfully written to the file descriptor.  This function should
  perform any tasks required to close the file descriptor and return a
  value less than zero in the case of any error.

\item {\bf {\tt context}} [optional] pointer to any additional information.
  Should the implementation of any of the above functions
  require access to other information, a pointer to this information
  can be stored in the {\tt context} member variable and retrieved by
  casting this member inside the function.

\end{enumerate}

\chapter{Testing the DADA Software}
\label{app:software}

This chapter describes the various programs that have been designed to
facilitate testing and development of the DADA software

\section{Primary Write Client Demonstration, {\tt dada\_pwc\_demo}}

The Primary Write Client (PWC) Demonstration program, {\tt
dada\_pwc\_demo}, implements an example PWC interface.  It does not
actually acquire any data and therefore can be run on any computer.
This program has two modes of operation: free and locked.

In {\em free} mode, {\tt dada\_pwc\_demo} does not connect to the
Header and Data Blocks; therefore, it is not necessary to create the
shared memory resources and run a Primary Read Client program.  This
mode is most useful when testing the command interface and state
machine of the Primary Write Client.  To run in {\em free} mode,
simply type
\begin{verbatim}
dada_pwc_demo
\end{verbatim}

In {\em locked} mode, {\tt dada\_pwc\_demo} connects to the Header and
Data Blocks; therefore, it is necessary to first create the shared
memory resources and also run a Primary Read Client program, such as
{\tt dada\_dbdisk}.  This mode is most useful when testing the
interface between Primary Write Client, Header and Data Blocks, and
Primary Read Client software.  To run in {\em locked} mode, for
example
\begin{verbatim}
dada_db -d         # destroy existing shared memory resources
dada_db            # create new shared memory resources
dada_pwc_demo -l   # run in locked mode
\end{verbatim}
The first step is particularly useful when debugging.
In another window on the same machine, you might also run
\begin{verbatim}
dada_dbdisk -WD /tmp
\end{verbatim}

To connect with the PWC demonstration program and begin issuing
commands, simply run
\begin{verbatim}
telnet localhost 56026
\end{verbatim}
or replace {\tt localhost} with the name of the machine on which the
program is running.

\subsection{Primary Write Client Command, {\tt dada\_pwc\_command}}

It is also possible to control one or more instance of {\tt
dada\_pwc\_demo} using the Primary Write Client Command program, {\tt
dada\_pwc\_command}.  By default, {\tt dada\_pwc\_command} uses the
same port number (56026) as {\tt dada\_pwc\_demo}.  Therefore, if you
are running both programs on the same machine, it will be necessary to
specify a different port number for the command interface; e.g.
\begin{verbatim}
dada_pwc_command -p 20013
\end{verbatim}

The PWC command program must be configured as described in ???;
it also requires the use of specification files to prepare for recording.


\end{document}
