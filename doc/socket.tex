\chapter{Socket Communications}

\section{Examples}
\subsection{Example Server}

\begin{verbatim}
#include "sock.h"

[...]

  char hostname [100];
  int port = 20013;

  /* Ask for the fully qualified hostname ... */
  sock_getname (hostname, 100, 1);
  /* ... or, ask for the IP address */
  sock_getname (hostname, 100, 0);

  int sfd = sock_create (&port);
  if (sfd < 0)
    perror ("Error creating socket");

  fprintf (stderr, "listening on %s %d\n", hostname, port);
  int cfd = sock_accept (sfd);
  if (cfd < 0)
    perror ("Error accepting connection");
\end{verbatim}
The open file descriptor returned by {\tt sock\_accept} may be used
in calls to the standard C I/O routines, {\tt read} and {\tt write}
as well as {\tt send} and {\tt recv}.  Furthermore, the open file
descriptor can be converted into a stream by calling {\tt fdopen}.
It is important to note that a stream should be opened for only read
or write access, never both; e.g.
\begin{verbatim}
  /* two separate file streams are required for reading and writing */
  FILE* sockin = fdopen (cfd, "r");
  FILE* sockout = fdopen (cfd, "w");

  /* set the socket output to be line-buffered */
  setvbuf (sockout, 0, _IOLBF, 0);
\end{verbatim}
The server can now read from {\tt sockin} using standard C stream I/O
routines such as {\tt fscanf}, {\tt fread}, and {\tt fgets}.  It can
also write to {\tt sockout} using {\tt fprintf} and {\tt fwrite}.

\subsection{Example Client}

\begin{verbatim}
#include "sock.h"

[...]

  char* hostname = "puma2.tms.nl";
  int port = 20013;

  /* Connect to the specified host and port */
  int cfd = sock_open (hostname, port);
  if (cfd < 0)
    perror ("Error opening socket");

  fprintf (stderr, "connected to %s %d\n", hostname, port);
\end{verbatim}
As with the server, the socket file descriptor may be accessed using
standard C stream I/O routines by calling {\tt fdopen}.
