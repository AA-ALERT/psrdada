\chapter{Configuration and Scheduling Software}

PuMa-II configuration will depend upon the objectives of the
experiment and may change from observation to observation.

\section{Configuration}

The configuration software will set up various operational parameters,
including:

\begin{itemize}
\item DMA buffer size
\item Data Block buffer size
\item network packet size
\item number of Primary nodes
\item number of Secondary nodes
\item assignment of Secondary to Primary node
\item operational mode: offline, simultaneous, or real-time
\item overlap required between Secondary nodes
\end{itemize}

\section{Scheduling}

The scheduling software will work with a data base in order to decide
the best configuration parameters, based on the observed target or
source name.
