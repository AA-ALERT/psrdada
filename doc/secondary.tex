\chapter{Secondary Unit}

The Secondary units will contain a large amount of RAM and some
moderate data storage medium, such as a magnetic tape device or hard
drive.  Secondary units will receive large blocks of digitized data
from the Primary Unit via high-speed interconnect, and process the
data within a flexible system of handshaking.  Although any number of
Data Client processes may be run on the Secondary units, this paper
will provide details for two specific clients: a Data Storage Client,
and a Data Reduction Client.

\section{Data Flow Control Software}

Data Flow Control software running on the Secondary units will
establish and maintain high-bandwidth data communication channels with
a single Primary Unit.  Data received via this communication channel
will be copied to the Data Block.  By a handshaking protocol described
in Section~\ref{sec:data_block}, the Data Flow Control Software will
cooperate with Data Client Software and ensure that all received data
is processed (storage or reduction).

\section{Data Client Software}

Data Client software running on the Secondary units will read the
digitized data from the Data Block shared memory and operate on this
data.  Upon completion of a sub-block of data, it may be flagged as
processed, allowing the Data Flow Control software to write over this
sub-block.  This handshaking protocol is similar to that used by the
EDT DMA driver.

\subsection{Data Storage Client: {\tt db2disk}}

Write a block to disk, flag as processed.

\subsection{Data Reduction Client: {\tt dspsr}}

Process a block of data and flag as processed or process a file on
disk and remove it.

\section{Data Block}
\label{sec:data_block}

The Data Block will be allocated as a shared memory resource.  It will
consist of a primary header followed by a number of sub-blocks.  Each
sub-block will have a corresponding variable to indicate the state of
the block: empty, data valid, or data processed.

All processes may read from the Data Block shared memory, but only the
Data Flow Control process may write to it.  In addition, only one Data
Client will be given the permission to set the ``data processed'' flag
of the sub-blocks.

Only the data from one observation can be held in the Data Block at
one time.  The relevant observation information (such as bandwidth,
centre frequency, source, start time, etc.) is stored in the primary
header space of the Data Block.  Each sub-block with ``data valid''
will also have an associated byte count that may be used to calculate
the time offset from the start of the observation.

If the Data Flow Control software cannot obtain a ``data processed''
or ``empty'' sub-block, an overflow condition occurs, and overflow
handling routines must propagate an appropriate message to the Primary
unit.  Depending upon the mode of operation, this condition may be
interpreted as an error, or as a signal to move on to the next
Secondary unit in the data transmission queue.
