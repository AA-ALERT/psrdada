\chapter{Data Reduction Software}

\section{Operational Phases}

The data reduction software will evolve through three phases in the life
of the PuMa-II instrument.

\begin{itemize}
\item {\bf offline} processing: performed after the recording has finished
\item {\bf simultaneous} processing: performed during the recording
\item {\bf diskless} processing: as above, but completely in memory
\end{itemize}

\subsubsection{Offline Data Reduction}

In the initial version of PuMa-II, the data reduction software will
operate only on data files stored on local disk space, and only after
the recording has completed.  After the observations have been
completed, the data files from each Primary node will be farmed out to
the Secondary nodes, and {\tt dspsr} will be run on each data file, as
specified by the Configuration and Scheduling software.

\subsubsection{Simultaneous Data Reduction}

In the next stage of development, the data reduction software will be
run during the observation.  In this mode, data are sent directly to
Secondary nodes and written to disk.  As before, {\tt dspsr} will be
run on the data files.

\subsubsection{Diskless Data Reduction}

In the final stage of PuMa-II evolution, the data reduction software
will be able to keep up completely with the flow of data on the
Primary nodes.  In this case, the data is never written to disk, and
{\tt dspsr} will operate as a Read Client directly connected to the
Data Block.  Another Read Client may be written that will farm data
reduction tasks to a grid computing facility.

\section{Configuration and Scheduling}

The data processing parameters will be specified by the Configuration
and Scheduling software, which will schedule and record all data
reduction operations.
