\chapter{Control Interface Software}

The Control Interface software will implement the single instrument
look and feel of PuMa-II.  A single process, {\tt puma2}, will:

\begin{itemize}
\item connect to the Command and Monitoring interface of required processes
\item issue commands as necessary to the required processes
\item collate monitoring information from the cluster nodes
\item provide Command and Monitoring connections to outside world
\item start and stop PuMa-II processes on any node as required
\end{itemize}

The number and variety of Data Flow Control Command and Monitoring
connections required will depend upon the amount of flexibility
desired. For example, if a configuration parameter such as the size of
the data blocks written to Secondary node (or file) must change from
observation to observation, then it will be necessary to communicate
these changes to the {\tt db2nic} or {\tt db2disk} processes on each
Primary node.  It will also be necessary to associate an observation
ID with each configuration.  Because of the large file storage space
on the Primary nodes, the {\tt db2nic} process may significantly lag
(in time) the primary control process.  Therefore, the configuration
commands that it receives would have to be queued.  Even with only the
Data Block buffer, the Data Client process could be busy when
configuration commands for the next observation are sent.

In addition to maintaining the required control connections with
processes on each of the Primary nodes, the Control Interface software
will also present a Command and Monitoring connection to the outside
world.  To this connection, an operator can connect to issue commands
and check on the status of the instrument/observation.  As before,
multiple communication channels may be connected, but only one will be
able to send control commands.  The operator will connect using a
textual user interface program, {\tt puma2tui}, which will present all
information from the Control Interface in an organized manner,
possibly employing the curses terminal control library.

Another process, {\tt puma2tms}, will be written to provide automated
control of the PuMa-II instrument by the WSRT TMS.  Commands sent by
TMS will be received by {\tt puma2tms}, converted to text commands,
and sent to {\tt puma2} through the operator interface.

The Control Interface software will enable complete initialization of
the PuMa-II instrument through a single command.  That is, on any or
all nodes, it will be able to start and stop various processes, create
and destroy the Data Block shared memory and semaphore resources, and
perform whatever other tasks prove useful in the initialization and
configuration of the PuMa-II instrument.

