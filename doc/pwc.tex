\newpage
\section{Primary Write Client Command Interface}
\label{sec:pwc}

This section describes the behaviour of the Write Client software that
will run on each of the Primary nodes, known as the Primary Write
Client (PWC) software.  In the case of PuMa-II, the PWC is 
{\tt puma2\_dmadb}.

\subsection{Operational States}

The PWC has four main states of operation:
\begin{itemize}
\item {\bf idle} waiting for configuration parameters
\item {\bf prepared} configuration parameters received; waiting for start
\item {\bf recording invalid} recording data in over-write mode
\item {\bf recording valid} recording data in lock-step mode
\end{itemize}

\subsubsection{Idle State}

In the idle state, the PWC sleeps until configuration parameters are
sent from the control software.  All configuration paramters are sent
in a single ASCII header.  This header is copied to the Header Block,
and the PWC enters the {\bf prepared} state.

\subsubsection{Prepared State}

In the prepared state, the PWC sleeps until a start command is sent
from the control software.  There are three different start commands
that can be received in this state:

\begin{itemize}
\item {\tt\bf INV\_START} enter the {\bf recording invalid} state
\item {\tt\bf START} enter the {\bf recording valid} state
\item {\tt\bf START $\langle duration\rangle$} same as {\tt START}, record
	for the duraction specified in either {\it seconds}, {\it samples},
	or {\it HH:MM:SS}.
\end{itemize}
For each of the above commands, the PWC will enter the specified state
at the next available opportunity (for PuMa-II, on the next {\tt
SYSTICK}).

\subsubsection{Recording Invalid State}

In this state, the PWC software clocks data into the Data Block but
does not flag the data as valid.  The PWC will overwrite the data in
each sub-block, and will remain in this state until one of the
following commands is received:

\begin{itemize}
\item {\tt\bf STOP} enter the {\bf idle} state immediately
\item {\tt\bf VAL\_START YYYY-MM-DD-hh:mm:ss} raise the valid data flag
	at the specified UTC time in the data stream and enter the {\bf
	recording valid} state
\item {\tt\bf VAL\_START YYYY-MM-DD-hh:mm:ss $\langle duration\rangle$} same 
	as {\tt VAL\_START}, record for the duraction specified in either
	{\it seconds}, {\it samples}, or {\it HH:MM:SS}.
\end{itemize}

Note that the UTC time specified in the first argument to {\tt
VAL\_START} may be any time in the future.  If it is in the past, then
the difference between the specified UTC and the present cannot be
greater than the amount of time corresponding to the length of the
Data Block.

\subsubsection{Recording Valid State}

In this state, the PWC software clocks data into the Data Block, flags
the data as valid, and will not overwrite a sub-block until it has
been flagged as cleared.  The PWC will remain in this state until one
of the following commands is received:

\begin{itemize}
\item {\tt\bf STOP} enter the {\bf idle} state immediately
\item {\tt\bf STOP YYYY-MM-DD-hh:mm:ss} enter the {\bf idle} state
	at the specified time
\item {\tt\bf VAL\_STOP YYYY-MM-DD-hh:mm:ss} raise the end of data flag
	at the specified UTC time in the data stream and enter the {\bf
	recording invalid} state
\end{itemize}

Note that the UTC time specified in the first argument to both {\tt STOP}
and {\tt VAL\_STOP} {\bf must} be in the future.

\subsection{Top Down Description}

The following describes the behaviour of the Primary Write Client

\begin{itemize}
\item Initialization
	\begin{itemize}
	\item read configuration file
	\item parse command line options
	\item initialize DMA and PiC cards
	\item connect to Data and Header Blocks
	\item open a port and listen for command connection
	\end{itemize}
\item Main Loop
	\begin{itemize}
	\item Idle State
		\begin{itemize}
		\item wait for configuration
		\item set configuration
		\item enter {\bf prepared} state
		\end{itemize}
	\item Prepared State
		\begin{itemize}
		\item wait for a command
		\item if command={\tt *START}, enter recording state
		\item if command={\tt STOP}, return to {\bf idle} state
		\end{itemize}
	\item Recording (Valid or Invalid) State
		\begin{itemize}
		\item check for a command
		\item if command={\tt STOP}, return to {\bf idle} state
		\item if state={\tt recording invalid} and command={\tt VAL\_START}, \\ enter the {\bf recording valid} state
		\end{itemize}
	\end{itemize}
\item Shutdown
	\begin{itemize}
	\item close command connection
	\item disconnect from Data and Header Blocks
	\end{itemize}
\end{itemize}



