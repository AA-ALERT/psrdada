\chapter{Operational Logging}

The various messages produced by the data acquisition software must be
logged and/or communicated to possibly more than one listener.  Therefore,
all messages will be sent using the {\tt multilog} API.  A multilog session
is opened by calling
\begin{verbatim}
multilog_t* multilog_open (const char* program_name, char syslog);
\end{verbatim}

\begin{itemize}
\item {\tt program\_name} the name of the program will precede each log message

\item {\tt syslog} if non-zero, all messages are cc'd to syslog
\end{itemize}
and closed by calling
\begin{verbatim}
int multilog_close (multilog_t* log);
\end{verbatim}

\begin{itemize}
\item {\tt log} pointer to an open multilog session
\end{itemize}
Once opened, file streams may be added by calling
\begin{verbatim}
int multilog_add (multilog_t* log, FILE* fptr);
\end{verbatim}

\begin{itemize}
\item {\tt log} pointer to an open multilog session

\item {\tt fptr} pointer to an open file stream
\end{itemize}
Messages are written to all file streams (and syslog, if enabled) by calling
\begin{verbatim}
int multilog (multilog_t* log, int priority, const char* format, ...);
\end{verbatim}

\begin{itemize}
\item {\tt log} pointer to an open multilog session

\item {\tt priority} a {\tt syslog} priority

\item {\tt format} an fprintf-style formatting string

\item {\tt ...} the value(s) to be printed according to the format
\end{itemize}

Output messages are assigned a priority as described in the man page
for the standard C {\tt syslog} utility.

\section{Example}

\begin{verbatim}
#include "multilog.h"

[...]

  /* open a connection to syslogd using the standard C facility */
  openlog ("dada_db2disk", LOG_CONS, LOG_USER);

  /* open a multilog session that will use syslog */
  multilog_t* log = multilog_open (1);

  /* copy all messages to the standard error */
  multilog_add (log, stderr);

  /* write a message */
  char* world_name = "Earth";
  int world_number = 1;

  multilog (log, LOG_INFO, "Hello %s %d", world_name, world_number);
\end{verbatim}
